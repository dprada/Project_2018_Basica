\documentclass[12pt,letterpaper]{article} %report,article?
\usepackage[letterpaper,lmargin=3.0cm,rmargin=3.0cm,tmargin=2.5cm,bmargin=2.5cm,headheight=12pt]{geometry}

\usepackage[utf8]{inputenc}
\usepackage[spanish, es-tabla]{babel}
\usepackage{color}
\usepackage{colortbl}
\usepackage{amsmath}
\usepackage{tocloft}
\usepackage{authblk}
\usepackage{fancyhdr}
\usepackage{soul}
\usepackage{cite}

\usepackage{helvet}
\renewcommand{\familydefault}{\sfdefault}

\usepackage{multicol}
\usepackage{etoolbox}
\usepackage{relsize}
\patchcmd{\thebibliography}
  {\list}
  {\begin{multicols}{2}\smaller\list}
  {}
  {}
\appto{\endthebibliography}{\end{multicols}}

\newcommand{\red}[1]{\textcolor{red}{\textbf{#1}}}
\newcommand{\blue}[1]{\textcolor{blue}{#1}}
\newcommand{\tout}[1]{\red{\textst{#1}}}

\renewcommand{\baselinestretch}{1.0}
\renewcommand\cftsecleader{\cftdotfill{\cftdotsep}}
\renewcommand{\labelitemi}{$\bullet$}
\renewcommand{\labelitemii}{$\circ$}
\renewcommand{\labelitemiii}{$\textendash$}

\pagestyle{fancy}
\fancyhead{}
\fancyfoot{}
% \fancyhead[RO,RE]{Drogabilidad Oncoproteínas EBV / DPG-HIMFG}
\fancyhead[R]{Drogabilidad Oncoproteínas EBV / DPG-HIMFG}
\fancyfoot[C]{\large \thepage}


\title{Evaluación computacional del potencial de modulación por fármacos (drogabilidad) de oncoproteínas del
  virus de Epstein-Barr para el tratamiento del cáncer. \\
  \vspace*{0.5\baselineskip}
  \line(1,0){250}\\
  \small{Tipo de apoyo: Investigador Joven} \\
  \small{Area: II. Biología y Química}\\
  \vspace*{0.5\baselineskip}
  \small{Responsable técnico: Diego Prada Gracia} \\
  \small{Institución de adscripción: Hospital Infantil de México Federico Gómez}\\
  \line(1,0){250}
}
\author[]{}
% \affil{Hospital Infantil de México Federico Gómez}
\date{}


\begin{document}
 
\maketitle

\section{Resumen}

El oncovirus de Epstein-Barr (EBV) es uno de los virus más extendidos entre los
humanos. Se estima que más del 90$\%$ de los adultos en el planeta están
infectados por EBV. Este virus tiene un ciclo de vida bifásico con una etapa
latente y otra lítica. Durante la etapa latente el virus permanece en forma de
plásmido, doble hélice de ADN cerrada circular, replicándose junto con la célula
hospedera que mayoritariamente es un linfocito B. En esta fase el virus expresa
proteínas con potencial oncogénico, inflamatorio y autoinmune; pudiendo
desencadenar la transformación del linfocito B en célula tumoral como ocurre en
cánceres como el linfoma de Burkitt, el linfoma de Hodgkin, el carcinoma
nasofaríngeo o el carcinoma gástrico. Los productos expresados por el virus en
esta fase están bien determinados, especialmente las proteínas virales. Sin
embargo no existe a la fecha tratamiento antiviral capaz de inhibir a EBV
latente. En la última década, tras el esclarecimiento
del rol de algunas de las proteínas de EBV, se han comenzado a sofisticar los
proyectos de descubrimiento de potenciales inhibidores que tradicionalmente se
basaban en campañas de cribado masivo experimental; con la implementación del
diseño racional asistido por computadora. Sin embargo estos acercamientos,
reportados tras la resolución de la estructura de varias de estas proteínas, se
han realizado únicamente mediante el diseño basado en estructura. No aparece en
la literatura ningún estudio que, por medio de modelos físicos de sistemas
biomoleculares integrados por computadora, evalúe la potencial inhibición
o modulación de la función de estas proteínas mediante la interacción con una
molécula diseñada ad hoc (drogabilidad). El presente proyecto que aquí se
presenta pretende ser pionero en la caracterización de los mecanismos que
determinan la drogabilidad de dichas proteínas con el propósito de
ser capaces de guiar el diseño racional de antivirales de EBV.


\section{Antecedentes}

Diversos factores, genéticos y epigenéticos, pueden desencadenar la
proliferación anómala de células transformando su ciclo de vida. Uno de estos
agentes desencadenantes son los virus oncogénicos \cite{Gupta2014,Damania2015}. Estos virus,
si bien desarrollan diferentes estrategias, tienen en común que producen
elementos ajenos a la célula como proteínas o fragmentos de RNA con la capacidad
de transformarla en tumoral. Alrededor del 20$\%$ de los cánceres humanos tienen
alguno de los ADN oncovirus desempeñando un papel decisivo en su patogénesis
\cite{Parkin2006,Dayaram2008}. Este es el caso del virus ADN oncogénico herpesvirus humano
4, o como comúnmente se denomina honrando a sus descubridores: virus de
Epstein-Barr (EBV).

EBV es un gamma herpesvirus B linfotrópico del que somos portadores la gran
mayoría de nosotros. Se estima que más del 90$\%$ de la población adulta mundial
está infectada por este oncovirus \cite{Cohen2000}. Si atendemos a la población
infantil existen diferencias geográficas debido a factores socio-económicos, demográficos y
culturales. En países en vías de desarrollo el contagio suele producirse a edades muy
tempranas, en la niñez, mientras que en sociedades desarrolladas dicho contagio
se retrasa mayoritariamente a la adolescencia \cite{Dowd2013}.

El mecanismo de infección y alojamiento en el hospedero adoptado por EBV tiene
particularidades que lo hacen especialmente exitoso y que explican que sea uno
de los virus más frecuentemente encontrado en humanos
\cite{Lieberman2014,Gupta2014,Damania2015,Klein2015,Minarovits2017}. Este
oncovirus se transmite entre individuos a través de la saliva, siendo las
células epiteliales del tubo digestivo superior la puerta de entrada más
frecuente. Una vez en el hospedero, EBV infecta preferentemente a linfocitos B
en los que puede ``residir'' replicándose con ellos durante toda la vida del
individuo infectado sin manifestar ni desencadenar ninguna patología, de manera
asintomática. El ADN viral se conserva como episoma expresando una cantidad
mínima de productos cuya función principal es desarrollar los mecanismos que
aseguran su conservación, transcripción y replicación. Replicación que se
produce una vez por ciclo celular mediante el acoplamiento físico del episoma al
cromosoma de la célula hospedera en su metafase. Esta etapa del ciclo de
infección del virus es conocida como ``fase de latencia'' y en ella se pueden
distinguir a su vez varias subfases (0, I, II y III) caracterizadas por la
presencia de algunas de las proteínas virales. Tras el contagio el virus adopta
la fase de latencia III en la que se expresan los antígenos nucleares EBNA (del
inglés Epstein-Barr Nuclear Antigen): EBNA1, EBNA2, EBNA3A, EBNA3B, EBNA3C y
EBNA-LP; además de las proteínas de membrana LMP1, LMP2a y LMP2b, y varios
miRNAs \cite{Kang2015b}. En la latencia II, I y 0 únicamente algunas de estas
proteínas se expresan, siendo EBNA1 la única presente en todas las fases. Es por
este motivo que EBNA1 ha atraído la atención de todo el que pretende desentrañar
el mecanismo molecular que posibilita que EBV permanezca tanto tiempo de manera
silente en el hospedero sin levantar ninguna alarma antigénica
\cite{Israel2003,Jiang2017,Lima2013a,Daskalogianni2015,Noh2016b,Gianti2016,Kosowicz2017}.
La importancia clínica de este ciclo latente reside en que los productos
expresados por los genes de EBV tienen capacidades oncogénicas, inflamatorias y
autoinmunes pudiendo transformar la célula hospedera infectada en tumoral
\cite{Humme2003,Klein2015}. Desafortunadamente en la actualidad no existen
fármacos con potencial inhibidor antiviral para EBV latente \cite{Kempkes2015}.

Tras la fase latente, y sin que haya sido esclarecido todavía el mecanismo que
desencadena esta transición, el virus entra en fase lítica. En esta nueva fase,
a diferencia de la anterior, EBV produce viriones con la participación de ADN
polimerasa del hospedero con el objeto de infectar otras células o ser
transmitidos por saliva para la infección de nuevos individuos. El mecanismo
desarrollado por el virus en esta fase es completamente distinto al anterior
implicando la expresión de nuevos productos. Las primeras proteínas expresadas,
y a las que se les adjudica un papel decisivo en el paso latente-lítico por su
participación en la expresión de posteriores productos, son las denominadas
BZLF1 y BRLF1. Posteriormente se expresan otras 42 proteínas
  líticas \cite{Traylen2015}. Para esta fase sí contamos con antivirales
capaces de atacar al virus con cierta eficacia. Es por este motivo que una de
las estrategias mas exploradas para tratar la infección de EBV es la inducción
de su fase lítica \cite{Lima2013a}.

Se estima que EBV se encuentra asociado al origen del 2$\%$ de los cánceres
humanos \cite{Parkin2006,Cohen2011,ZurHausen2015}. Este oncovirus se encuentra
en el origen a varios tipos de linfomas de linfocitos B y T como los linfomas de
Burkitt o Hodgkin, así como los carcinomas nasofaríngeo y de estómago, y
sarcomas de músculo liso. También se encuentra asociado a enfermedades
autoinmunes e inflamatorias como mononucleosis infecciosa, síndrome crónico de
EBV y síndrome histiocítico, esclerosis múltiple o lupus eritematoso sistémico
entre otras \cite{Parkin2006,Cohen2011,Young2016,Draborg2013,Niller2008}. Todo
esto convierte al virus EBV en una de las principales infecciones a controlar en
pacientes inmunodeprimidos evidenciando también la importancia clínica de
encontrar moléculas inhibidoras de sus oncoproteínas preferencialmente en su
fase latente.

Desde antes de resolver la estructura cristalizada, sola o en su interacción con
ADN \cite{Bochkarev1995,Bochkarev1996c}, la proteína EBNA1 ha sido la principal
diana del diseño de fármacos \cite{Orlowski1990,Henderson1982}. Además de ser la
única proteína expresada en cualquier punto del ciclo de vida bifásico del
virus, la polivalencia de sus mecanismos moleculares la convierte en la piedra
angular sobre la que descansa la supervivencia de EBV en latencia: realiza
funciones de persistencia, replicación y transcripción del genoma. Para realizar
estas funciones EBNA1 reparte sus 641 aminoácidos en distintos dominios con
distinto rol. Así por ejemplo, este antígeno nuclear es capaz de unirse en
distintos puntos al episoma vírico en forma de homodímero mediante su dominio de
interacción a ADN, bien para contribuir en su interacción con la maquinaria
proteica del hospedero a su transcripción, bien para hacer de enlace físico
entre el episoma y el material cromosómico de la célula \cite{Bochkarev1996c}.
Otro dominio cuyo rol es reseñable es el dominio de repetición de pares de
aminoácidos arginina-alanina. Este dominio ha demostrado ser uno de los
responsables de que la célula infectada en fase latente pase desapercibida
frente a linfocitos T citotóxicos CD8 \cite{Lee2004}. Además, interacciona con
proteosomas e inhibe la proteólisis de dicha proteína \cite{Dantuma2000}.

Tras EBNA1, otra proteína que destaca por acumular numerosos ensayos para
determinar la naturaleza de sus funciones es BZLF1. La proteína BZLF1, cuya
estructura fue resuelta en su interacción con ADN en 2006 \cite{Petosa2006}, es
un regulador transcripcional que aparece como el primer producto que EBV expresa
en su fase lítica. Este hecho la relaciona de manera reconocida en la
literatura como el posible agente interruptor que
desencadena el resto de acontecimientos de la fase lítica
\cite{Petosa2006,Daskalogianni2015,Hartlage2015,Lima2013a}.
\\

En las últimas décadas se han llevado acabo campañas de cribado masivo de
moléculas, inicialmente experimentales, en la búsqueda de un antiviral efectivo
especialmente para la fase latente
\cite{Henderson1982,Orlowski1990,Li2010a,Li2010b,Lima2013a}. Aparece en la
literatura reciente un no despreciable número de moléculas encontradas \emph{de
  novo} al que se unen aquellos compuestos como antivirales de alto espectro
cuyo efecto ha sido probado en EBV
\cite{Gershburg2005,Li2010a,Clercq2002,Hornig2014}. Casi todas estas moléculas
se reportan con problemas de biodisponibilidad, toxicidad o ineficiencia. Sobre
muchas de ellas se desconoce su molécula blanco y por lo tanto las
características del sitio de unión como para poder emprender proyectos de
optimización. El resultado de esta situación es que actualmente no contamos con
moléculas capaces de actuar como antivirales en la fase latente, y los intentos
experimentales de descubrimiento de moléculas inhibidoras no han contribuido a
acumular conocimiento sobre los mecanismos de drogabilidad de las proteínas de
EBV \cite{Gianti2016}. En cuanto a los intentos de diseño racional computacional
que aparecen en la literatura, estos están todavía en fase incipiente y no
abundan. La mayoría de estos intentos reportados hacen uso del diseño basado en
estructura partiendo directamente de la caracterización obtenida del cristal:
obviando las modificaciones de su estructura en un proceso de relajación o
ignorando cualquier cambio conformacional que pueda estar en el origen del
mecanismo a escala atómica de su función
\cite{Li2010a,Li2010b,Hornig2014,Gianti2016,Young2016,Messick2017,Jiang2017}.
Solamente un estudio aparece apoyado en la simulación de dinámica molecular para
el diseño de un inhibidor de EBNA1. Se trata de un experimento de simulación
``fuerza bruta'' durante un tiempo de 200 ns \cite{Jiang2017}, a todas luces
insuficiente para determinar la estabilidad de la unión y la naturaleza de las
interacciones atómicas que más contribuyen entálpica y entrópicamente a la
energía libre de unión. Otro elemento característico y común a los intentos de
diseño racional de inhibidores sobre las proteínas de EBV es el abordaje del
filtrado de quimiotecas únicamente mediante su interacción o docking de proteína
\cite{Ferreira2015,Lavecchia2013}. Estos métodos, lejos de evaluar la física de
todos los grados de libertad del sistema obteniendo magnitudes físicas relativas
a la interacción, se apoyan en argumentos heurísticos para la rápida puntuación
de la idoneidad de la interacción. En algunos casos, esta puntuación está basada
en la evaluación de una energía potencial de interacción de un modelo de
biomoléculas reducido, introduciendo algunas interacciones electrostáticas de
cargas puntuales, interacción estérica o parametrización sencilla de puentes de
hidrógeno, por ejemplo. Sin embargo, estas aproximaciones nunca consideran
ninguna contribución entrópica y por lo tanto no estiman las energías libres de
unión de las distintas moléculas candidato, que son las que determinan
cuantitativamente las probabilidades de estancia de un sistema físico en un
determinado estado (en este caso ``acoplados'')
\red{\cite{Shirts2010a,Mortier2015}}. Tampoco de esta manera se pueden validar
sus predicciones dado que no hay magnitudes físico-químicas termodinámicas
-relativas o absolutas- y/o cinéticas de sus modelos que poder comparar con
experimentos in vitro \cite{Chaires2008}.

Existe una ausencia importante de estudios teórico computacionales que aborden,
termodinámica y cinéticamente, el conjunto de interacciones y cambios
estructurales a escala atómica que permitan determinar el potencial que tienen
ciertas proteínas de EBV, potenciales blancos para su inhibición como EBNA1 o
BZLF1, para ver alguna de sus funciones interrumpida o modulada mediante la
interacción con moléculas diseñadas ad-hoc (drogabilidad \cite{Barril2013}).


\section{Planteamiento del problema}

El oncovirus EBV se encuentra en el origen del mecanismo que desencadena
diversos tipos de cáncer como el linfoma de Hodgkin o el linfoma de Burkitt, así
como carcinomas nasofaríngeos o diversas enfermedades autoinmunes
\cite{Parkin2006,Cohen2011,Young2016,Draborg2013,Niller2008,ZurHausen2015}.
EBV tiene la capacidad de infectar linfocitos B y en su fase latente hospedarse
en la célula de manera asintomática sin más trascendencia hasta la muerte del
hospedero. Pero su carga viral, dependiendo del estadio
de la fase latente, puede presentar actividad autoinmune, inflamatoria y
oncogénica. El virus tiene la capacidad de transformar la célula infectada
tornándola inmortal
  \cite{Lieberman2014,Gupta2014,Damania2015,Klein2015,Minarovits2017}.

Las proteínas que EBV expresa en los distintos estadios de su ciclo de vida
bifásico (latente-lítico) están bien caracterizadas. Algunas de ellas, como es
el caso del antígeno nuclear de Epstein-Barr 1 (EBNA1) o la proteína BZLF1,
cuentan con la estructura resuelta de sus dominios de interacción con y sin
  ADN en ambos casos por difracción de rayos X 
  \red{\cite{Bochkarev1995,Bochkarev1996c,Petosa2006}}. \tout{Esta
  informaci\'on} Esto es
esencial para poder determinar, haciendo uso de modelos físicos computacionales,
los mecanismos a escala atómica que describen cómo estas proteínas llevan a cabo
su función mediante cambios conformacionales e interacciones con otras
biomoléculas. Esta información, no s\red{o}lamente estructural sino también
dinámica, permite revelar sitios de la proteína que convenientemente alterados
en sus propiedades físicas pueden modular su comportamiento
\cite{Shirts2010a,Mortier2015}. Este efecto puede producirse por la
interacción de dicha proteína con una molécula o péptido diseñados \emph{ad hoc}
con las propiedades adecuadas para inhibir su función, desestabilizar su
estructura, evitar la formación de una interfase, bloquear el sitio activo en el
caso de una enzima, u ocupar el sitio de unión en un receptor para desencadenar
un efecto agonista u antagonista modulado en ocasiones de manera alostérica. A
esta potencial modulación de la actividad de una proteína mediante moléculas
diseñadas para tal efecto se le conoce en el argot como ``drogabilidad'' \red{\cite{Barril2013}}. Y su
evaluación es el paso previo indispensable para iniciar un proyecto de diseño
y/u optimización de fármacos.

Este proyecto pretende centrarse en la evaluación de la drogabilidad de
proteínas del virus EBV con objeto de determinar regiones específicas en la
estructura de los blancos moleculares sobre las que comprobar la factibilidad del
diseño para su inhibición. De manera más concreta, y como hipótesis inicial, se
trabajará en la búsqueda de mecanismos de inhibición de EBNA1 y BZLF1 dejando
abierta la posibilidad de determinar la drogabilidad de cualquier otra proteína
con posible papel oncogénico.


\section{Justificación}

Se estima que el 2$\%$ de todos lo cánceres se deben a la infección por el
herpesvirus de Epstein-Barr (EBV)
\red{\cite{Parkin2006,Cohen2011,ZurHausen2015}}. Esta reportado que anualmente
aparecen 200.000 casos de cáncer (Linfoma de Burkitt, Linfoma de Hodgkin,
carcinoma nasofaríngeo, carcinoma gástrico) con evidencias de su asociación con
EBV \cite{Cohen2011}. Pero las dimensiones de la relevancia clínica de este
oncovirus son mayores si se tiene en cuenta que origina anualmente 125.000
nuevos casos de mononucleosis infecciosa sólo en Estados Unidos, o que aparecen
en el mundo aproximadamente un millón de nuevos casos de carcinoma gástrico cada
año \cite{Kempkes2015}. \tout{Espec\'ificamente, en el Hospital Infantil de
  M\'exico Federico G\'omez -instituci\'on vinculada al presente proyecto- la
  Unidad de Virolog\'ia y C\'ancer liderada por el Dr. Ezequiel Fuentes Panan\'a
  -colaborador del presente proyecto- tiene consolidada una linea de
  investigaci\'on sobre papel de dicho virus los linfomas pedi\'atricos
  atendidos en este hospital tiene contabilizado que el 70$\%$ de los linfomas
  pediatricos que se atienden en el hospital son causados por este virus.}
Muchas \tout{de las patolog\'ias desencadenadas por EBV, como el caso de dichos
  linfomas pedi\'atricos,} \red{de estas patologías} se originan con el virus en
su fase de latencia\tout{. Y} y no existen a día de hoy tratamiento antiviral o
fármaco inhibidor de EBV en dicha fase \red{\cite{Kempkes2015,Lima2013a,Young2016}}.
Por contra, sí existe tratamiento para la fase lítica donde el virus requiere de
la participación de la ADN polimerasa del hospedero \red{o de su cinasas
  \cite{Hornig2014}}. Pero dicho tratamiento sólo se puede aplicar cuando la
infección se encuentra en pleno apogeo, lo cual es inconveniente cuando tratamos
por ejemplo con pacientes inmunodeprimidos.

Adicionalmente es pertinente \tout{rese\~nar} \red{señalar} que es\red{te}
proyecto de investigación es necesario y oportuno dado que no se encuentra en la
literatura iniciativas de diseño racional de nuevas moléculas \red{potenciales
  inhibidoras de EBV} basadas en el
estudio computacional de la termodinámica y cinética de los procesos de
interacción entre biomoléculas y sus cambios conformacionales. \tout{La
  relevancia de este hecho se puede explicar de manera sencilla con el siguiente
  ejemplo. La prote\'ina EBNA1 es un blanco molecular muy atractivo cuando se
  trata de intentar inhibir a EBV en su fase latente, y la estructura de esta
  prote\'ina ha sido resuelta hace m\'as de 10 a\~nos. Sin embargo, el dise\~no
  racional reportado hasta ahora se ha limitado a considerar \'unicamente la
  estructura r\'igida cristalizada para la que se reporta, ya desde el
  art\'iculo original, que en el complejo prote\'ina-ADN se advierten posibles
  interacciones estabilizantes todav\'ia no formadas. Como si el complejo
  hubiera encontrado una estructura estable para cristalizar previa a su
  completa estabilizaci\'on en su uni\'on. Sin embargo, no aparecen trabajos en
  los que se haya tratado de predecir cuales son las conformaciones mas estables
  ni c\'omo es el proceso de acercamiento y uni\'on para determinar qu\'e
  interacciones juegan un papel m\'as relevante, y en qu\'e momento, para
  determinar m\'as acertadamente su posible drogabilidad.} \red{Este hecho
  justifica el importante componente que esta propuesta tiene de explorativa}.
\red{Dada la ausencia casi completa de estudios sistemáticos sobre la dinámica de estas
proteínas este proyecto debe tener como una de sus principales vocaciones la
acotación y distinción de qué blancos y en qué puntos son susceptibles ser
drogados \red{\cite{Gianti2016}}. Esto servirá para enfocar más acertadamente posteriores proyectos de
diseño y optimización racional de moléculas con potencial farmacológico como
antivirales para EBV, además de darnos la oportunidad de ubicar nuestro trabajo
y resultados como referencia en dicho campo.}

\tout{
Este hecho nos da la oportunidad de aprovechar el presente
proyecto , adem\'as de para iniciar una prometedora linea de investigaci\'on en la
nueva Unidad de Investigaci\'on en Biolog\'ia Computacional y Dise\~no de F\'armacos del
Hospital Infantil de M\'exico, para ubicar nuestro trabajo y resultados como
referencia en el campo del dise\~no de antivirales de EBV.}

\tout{
Por \'ultimo se hace necesario justificar el componente importante que este
proyecto tiene de explorativo. Dado el vac\'io que hay en estudios sistem\'aticos
sobre la din\'amica de estas prote\'inas, sus cambios conformacionales o sus
propiedades termodin\'amicas y cin\'eticas, este proyecto debe tener como una de sus
principales vocaciones la acotaci\'on y distinci\'on de qu\'e blancos y en qu\'e puntos
son susceptibles ser drogados. Esto servir\'a para enfocar m\'as acertadamente
posteriores proyectos de dise\~no y optimizaci\'on racional de mol\'eculas con
potencial farmacol\'ogico como antivirales para EBV.
}

\red{  
\section{Hipótesis}
}

El conjunto de oncoproteínas expresadas en las fases de latencia de EBV, como
EBNA1, o aquellas necesarias para su tránsito de la fase latente a la forma
lítica, como BZLF1, pueden constituir unos blancos moleculares idóneos para el
diseño de inhibidores con potencial farmacológico sobre aquellos tipos de cáncer
inducidos por EBV. Los modelos físicos de biomoléculas y su simulación pueden
aportar las claves termodinámicas y cinéticas necesarias para evaluar
eficientemente su potencial drogabilidad y avanzar en el diseño de antivirales
para EBV.
\\
\section{Objetivos}

El objetivo general de este proyecto está descrito con claridad desde su
hipótesis principal.

\subsection{Objetivo general}

Evaluar el potencial de modulación que mediante \red{la} interacción con pequeños
ligandos o péptidos puede presentar la función de proteínas constitutivas del
virus EVB como EBNA1 o BZFL1.


\subsection{Objetivos particulares}

\begin{itemize}

\item Detectar y evaluar nuevos sitios drogables \tout{principalmente} sobre proteínas cuya inhibición
  puede jugar un papel decisivo en el diseño de antivirales para EBV.

\item Validar hipótesis de mecanismos de acción a escala atómica de proteínas
  relevantes cuya función es decisiva para EBV tanto en la fase latente como
  lítica. \tout{Ejemplo: describir el proceso de uni\'on de EBNA1 al episoma de EBV
  calculando la diferencia de energ\'ia libre entre EBNA1 en solvataci\'on y EBNA1
  en su interacci\'on con DNA; ratificar o rechazar las hip\'otesis que aparecen en
  la literatura sobre el papel del homod\'imero, los mecanismos de selectividad
  sobre segmentos de pares de bases espec\'ificos y el efecto colectivo de uni\'on
  en sitios pr\'oximos como los presentados en la regi\'on denominada ``Dyad
  Simmetry''.}

\item Recopilar de la literatura \red{la} quimioteca de moléculas cuya posible
  actividad antiviral sobre EBV ya ha\red{ya} sido reportada aun sin conocer su
  posible mecanismo de acción. \tout{Construir el m\'inimo conjunto de
    farmac\'oforos que represente dicha quimioteca para comprobar su posible
    encaje sobre los sitios drogables detectados.} \red{Identificar además, mediante el
    mínimo conjunto de farmacóforos que la representen, el posible encaje de
    estos ligandos sobre los sitios drogables detectados.}
  
\item Diseñar, sin atender a criterios de reglas de Lipinski como la apropiada
  solubilidad, biodisponibilidad o la alta selectividad o baja toxicidad, un
  conjunto de moléculas o péptidos \red{cuya teórica} alta afinidad por los nuevos sitios
  drogables detectados \red{pueda en el futuro servir como base para el posible
    diseño de un inhibidor.} \tout{
  a lo largo del proyecto cuya constante de disociaci\'on, entropia y
  energ\'ia libre de uni\'on calculados en los modelos f\'isicos puedan ser en el
  futuro corroborados y validados mediante el uso de la microcalorimetr\'ia
  de titulaci\'on isoterma en una posible extensi\'on de este proyecto.}

\item \tout{Dise\~nar, sin atender a criterios de reglas de Lipinski como la apropiada
  solubilidad, biodisponibilidad o la alta selectividad o baja toxicidad, un
  conjunto de mol\'eculas o p\'eptidos con alta afinidad por los nuevos sitios
  drogables detectados a lo largo del proyecto cuyo efecto inhibitorio pueda ser
  comprobado y caracterizado de manera experimental mediante el uso de la
  microcalorimetr\'ia de titulaci\'on isoterma en una posible extensi\'on de este proyecto.}

\item \tout{Consolidar el arranque de una importante linea de investigaci\'on definiendo
  sobre criterios s\'olidos el comienzo del camino hacia el dise\~no y optimizaci\'on
  de antivirales de EBV para el tratamiento de los distintos tipos de cancer
  asociados a este virus.}
  
\item \tout{Dise\~nar mediante la evaluaci\'on de la posible drogabilidad de prote\'inas de
  EBV experimentos, como por ejemplo la inhibici\'on de interacciones con
  prote\'inas del hospedero, que arrojen evidencias sobre los mecanismos
  oncog\'enicos, inflamatorios o autoinmunes que la infecci\'on puede desencadenar.}

\end{itemize}

  


\section{Metodología}

Tres herramientas computacionales se emplearán adecuadamente según la naturaleza
y función de la proteína blanco para la evaluación de nuevos sitios drogables:
el docking sobre proteína rígida de librerías de compuestos para el cribado
virtual masivo inicial (High Throughput Screening)
\red{\cite{Ferreira2015,Lavecchia2013}}, la simulación de dinámica molecular a
todos los átomos con solvente expl\red{í}cito o impl\red{í}cito (según convenga)
\red{\cite{Mortier2015}} y los métodos de evaluación de energía libre,
determinación del paisaje conformacional de energía libre y métodos de
perturbación de energía libre \red{\cite{Shirts2010a}}. Estos abordajes serán
aplicados sobre distintas proteínas constitutivas de EBV teniendo como
principales candidatas el antígeno nuclear 1, EBNA1, única proteína presente en
todas las fases de latencia del virus, y BZLF1, regulador transcripcional
decisivo en el paso de la forma latente a lítica. Por ejemplo, en el caso de
proteínas cuya drogabilidad no ha sido previamente explorada, como BZLF1, se
comenzará aplicando abordajes clásicos heurísticos de filtrado de quimiotecas
mediante docking sobre proteína rígida para revelar posibles sitios de unión
\red{\cite{Ferreira2015}}; a la vez que paralelamente se lleva a cabo una búsqueda
racional de regiones de interfase con aquellas moléculas con las que
interaccionen en su actividad, como p53 o ADN en este caso, susceptibles de ser
empleadas como posibles sitios drogables \red{\cite{Barril2013}}. Por contra, en el caso de una proteína
como EBNA1 cuya interacción en su unión con ADN ha sido recientemente empleada
sin éxito para el diseño de inhibidores, se llevarán acabo estudios de dinámica
molecular para por primera vez caracterizar los cambios conformacionales en el
proceso de unión y las interacciones inter-atómicas más relevantes \red{\cite{Mortier2015}}, así como
estudios de perturbación de energía libre cuya aplicación por definición resulta
más precisa, aunque más costosa computacional\tout{e}mente, para la predicción
de constantes de afinidad molécula blanco-ligando \red{\cite{Shirts2010a,Perez2016}}.

Se hace necesario puntualizar que aunque estas metodologías emplean el uso
masivo de computadoras para la simulación de la física del sistema biomolecular,
estas no han de ser confundidas con las metodologías propias del campo de la
bioinformática. \tout{Es cierto que} Ambas disciplinas comparten el uso de
herramientas estadísticas para el análisis de datos como el análisis bayesiano
\red{\cite{Perez2015,Kasson2009}}, los modelos de Markov
\red{\cite{Chodera2014}}, el análisis multivariante \red{\cite{Doer2017}} o
la potente batería de herramientas propias de la física de sistemas complejos
como el uso de teoría de redes complejas o teoría de la información
\red{\cite{Prada-Gracia2009a}}. La metodología a usar aquí está fundamentada en
las leyes \red{de la} física que \tout{dominan la realidad del} \red{describen y
  predicen el comportamiento de las interacciones de biomoléculas a escala de
  energía del orden de $~K_BT$, tiempos del orden de 10$^{-12}$ segundos y
  distancias en el rango de 10$^{-10}$ metros, posibilitando} su simulación.
Dicha simulación \tout{, o integraci\'on,} se hace mediante el uso de
computadoras debido a la \tout{alta dimensionalidad del sistema:} alta complejidad de
las ecuaciones a resolver. \tout{Pero dichas ecuaciones pertenecen a modelos
  f\'isicos fundamentados para representar y predecir fen\'omenos naturales a
  escala de energ\'ia del orden de $/K_BT$, tiempos del orden de femtosegundos y
  distancias en el rango de 10$^{-10}$ metros.} Esta f\'isica está bien
consolidada\tout{, en especial la que a procesos cl'asicos en equilibrio se refiere,}
desde los trabajos de Newton, Einstein (descripción browniana), Fokker y Planck
o Boltzman,\tout{ etc...}, \red{en especial la que a procesos cláxicos en
  equilibrio se refiere.} Así por ejemplo los modelos que se emplearán de biomoléculas en
este proyecto cumplen con la descripción de procesos estocásticos, la física
atómica, la física cuántica, la ley de fluctuación-disipación, la termodinámica
clásica y la mecánica estadística, entre otros. \tout{Es posible que} Debido a
la dimensión de los sistemas a estudiar, \red{cuando} las magnitudes de
equilibrio termodinámico no puedan ser muestreadas convergentemente mediante las
técnicas de simulación habituales \tout{. Para determinar de manera econ\'omica
  las propiedades de altas barreras de energ\'ia libre} se hará uso de técnicas
de simulación de procesos de no equilibrio para inferir información de
equilibrio (desigualdad de Jarzinsky y teorema de Crooks) \red{\cite{Sandberg2015}}, así como de
estrategias de reescalado del sistema como la simulación con campos de fuerza
coarse-grained \red{\cite{Perilla2015}}.

Las herramientas de bioinformática (minería de datos) y química computacional
serán usadas también para dar soporte al proyecto en aspectos como \tout{. Por
  ejemplo, en }la búsqueda de interacciones proteína-proteína posiblemente
descritas ya en la literatura \red{\cite{Lehne2009}}.
 

\section{Plan de análisis de los datos}

Dado que se trata de un proyecto teórico el plan de análisis de datos, que no se
puede anticipar en su totalidad, está descrito impl\red{í}citamente en la
metodología. En el caso de las simulaciones de dinámica molecular se llevarán
acabo análisis basados en redes cinéticas \red{\cite{Prada-Gracia2009a}} y
modelos de estados de markov \red{\cite{Chodera2014}}. En las experiencias de
docking \red{\cite{Ferreira2015}} o perturbación de energía libre
\red{\cite{Shirts2010a}} se emplearán técnicas de clustering para la estimación
de la contribución entrópica además del cálculo entálpico \red{\cite{Du2016}}.
Además, se hará uso de la batería de herramientas de análisis mencionadas en la
sección anterior como el análisis bayesiano \red{\cite{Kasson2009}}, el análisis
multivariante \red{\cite{Doer2017}} o las redes complejas
\red{\cite{Prada-Gracia2009a}}. Adicionalmente, como en todo proyecto
físico-teórico cabe la posibilidad de diseñar, según el rumbo del proyecto,
análisis con fundamento físico y estadístico adecuados para responder a las
distintas preguntas puntuales que surjan en el transcurso \red{del proyecto}
\tout{de la investigaci\'on}.

\section{Limitaciones del estudio}

La principal limitación con la que cuenta el proyecto se debe a los
insuficientes recursos computacionales con los que se cuenta en la actualidad en
la Unidad de Biología Computacional y Diseño de Fármacos del Hospital Infantil
de México. La simulación de un sistema que incluye un segmento de ADN además de
una proteína de más de 600 aminoácidos, como EBNA1, debidamente solvatados en
presencia de cationes y aniones requiere de un poder computacional mayor al del
cluster de cinco nodos \red{del que dispone nuestra unidad}. Lo mismo sucede con
el cribado \tout{HTS} de una gran quimioteca considerando la proteína rígida, o
el cribado \tout{MTS} de pocos ligandos mediante evaluación de energía libre de
unión. Se requiere de la participación de un volumen mayor de recursos
computacionales. Además, la dotación económica que supuestamente se destina para
soportar el desarrollo de este proyecto de investigación no puede ser destinada
a la compra del equipo de laboratorio computacional necesario \tout{sin duda}
para \tout{trabajar} este tipo de proyectos de investigación.

Para superar esta limitación se buscará la manera de acceder a tiempo de
computación en alguno de los tres nodos de supercomputación del Laboratorio
Nacional de Computo de Alto Desempeño, LANCAD, de México.

En caso de no disponer del poder computacional adecuado se comprobará la validez
de ciertas aproximaciones al modelo físico de sistema biomolecular con el fin de
reducir su complejidad a costa de perder detalle fino en la descripción y poder
ser utilizadas con el propósito de alcanzar los objetivos descritos en esta
propuesta. Estas técnicas pueden ser el uso de solvente impl\red{í}cito
\red{\cite{Kleinjung2014}},\tout{ el uso de estrategias de escalado
  dimensional,} m\red{é}todos de integración de dinámica estocástica
\emph{biased} sobre el espacio reducido de grados de libertad mas lentos
\red{\cite{Wu2016}}, el diseño de experimentos computacionales de no equilibrio
\red{\cite{Sandberg2015}} o el uso de campos de fuerza coarse-grained
\red{\cite{Perilla2015}}.

\section{Consideraciones éticas}

Dentro de las distintas categorías de riesgo según los agentes, organismos o
población manejada en el estudio podemos catalogar el presente proyecto de
investigación de ``Sin Riesgo''. El proyecto es puramente teórico. No se
realizarán en ningún momento experimentos físicos -que no sean computacionales-,
químicos o biológicos. Tampoco el objeto de estudio es un segmento de población
humana. No procede por tanto para este proyecto hacer ninguna consideración
ética relevante o incluir carta de confidencialidad, compromiso de
confidencialidad de datos personales o protocolo de actuación frente a efectos
adversos. \emph{Ver Formatos Adjuntos.}

\section{Consideraciones de bioseguridad}

Al igual que en la sección anterior. Se puntualiza de nuevo que este proyecto de
investigación es puramente teórico. No se realizarán en ningún momento
experimentos físicos -que no sean computacionales-, químicos o biológicos. No se
manejarán muestras biológicas, agentes patógenos, material radiactivo o
productos tóxicos. No es pertinente entonces incluir ninguna consideración
específica relativa a bioseguridad.
\emph{Ver Formatos Adjuntos.}

\section{Anexos}

Se adjuntan al final de la propuesta y como anexos fuera de índice los
siguientes documentos: ``Solicitud de evaluación de proyecto'', ``Formato del
comité de ética en investigación'', ``Formato del comité de bioseguridad'',
``Formato PE-04'', ``Formato de distribución de asignación anual'' y
``Justificación de distribución de asignación anual''.

\section{Cronograma}

El desarrollo del proyecto de investigación aquí propuesto se llevará acabo en
el transcurso de 8 trimestres de la siguiente manera:

\begin{itemize}
\item \emph{RevLit}: Revisión de literatura, recolección de ligandos reportados
  con potencial antiviral y ejercicio de discusión, diseño y evaluación de las
  estrategias a adoptar.
	\item \emph{RepRev}: Escritura de artículo de revisión enfocado hacia el diseño
    racional asistido por computadora de inhibidores de EBV.
	\item \emph{EvalProts}: Evaluación de idoneidad de proteínas expresadas por
    EBV distintas a EBNA1 y BZLF1 para la evaluación de su drogabilidad.
	\item \emph{SimuEBNA1}: Experimentos computacionales de EBNA1. 
	\item \emph{AnalEBNA1}: Análisis de experimentos computacionales de EBNA1.
	\item \emph{SimuBZLF1}: Experimentos computacionales de BZLF1.
	\item \emph{AnalBZLF1}: Análisis de experimentos computacionales de BZLF1. 
	\item \emph{SimuOtras}: Simulación de posibles proteínas alternativas
    seleccionadas en la etapa \emph{EvalProts}.
	\item \emph{AnalOtras}: Análisis de experimentos computacionales de \emph{SimuOtras}.
	\item \emph{EvalLigsLit}: Evaluación sobre las proteínas abordadas y sus
    sitios drogables de la colección de ligandos obtenida en la etapa \emph{RevLit}.
	\item \emph{RepRes1}: Escritura de comunicación de lo aprendido con EBNA1.
	\item \emph{RepRes2}: Escritura de comunicación de lo aprendido con BZLF1.
	\item \emph{RepRes3}: Escritura de comunicación de lo aprendido con las
    posibles proteínas seleccionadas en la etapa \emph{EvalProts}.
  \end{itemize}
  
\definecolor{midgray}{gray}{.5}
\begin{table}[!htbp]
	\centering
		\begin{tabular}{|c|c|c|c|c|c|c|c|c|}
      \hline
      &\multicolumn{4}{c|}{Año 1}&\multicolumn{4}{c|}{Año 2}\\
      \hline
		&1&2&3&4&5&6&7&8\\
		\hline
		\emph{RevLit}&\cellcolor{midgray}&&&&&&&\\
		\hline
		\emph{RepRev}&&\cellcolor{midgray}&&&&&&\\
		\hline	
		\emph{EvalProts}&&\cellcolor{midgray}&&&&&&\\
		\hline			
		\emph{SimuEBNA1}&&&\cellcolor{midgray}&\cellcolor{midgray}&&&&\\
		\hline	
		\emph{AnalEBNA1}&&&&\cellcolor{midgray}&\cellcolor{midgray}&&&\\
		\hline
		\emph{SimuBZLF1}&&&&\cellcolor{midgray}&\cellcolor{midgray}&&&\\
		\hline	
		\emph{AnalBZLF1}&&&&&\cellcolor{midgray}&\cellcolor{midgray}&&\\
		\hline	
		\emph{SimuOtras}&&&&&\cellcolor{midgray}&\cellcolor{midgray}&&\\
		\hline	
		\emph{AnalOtras}&&&&&&\cellcolor{midgray}&\cellcolor{midgray}&\\
		\hline	
		\emph{EvalLigsLit}&&&&&&&&\cellcolor{midgray}\\
		\hline	
		\emph{RepRes1}&&&&&\cellcolor{midgray}&\cellcolor{midgray}&&\\
		\hline	
		\emph{RepRes2}&&&&&&\cellcolor{midgray}&\cellcolor{midgray}&\\
		\hline	
		\emph{RepRes3}&&&&&&&\cellcolor{midgray}&\cellcolor{midgray}\\
		\hline	
		\end{tabular}
\end{table}

El programa es ambicioso pero su estructura permite el alcance de resultados
intermedios independientes. El cumplimiento de 2/3 de los resultados planteados
se puede considerar un desarrollo exitoso.

\section{Productos a obtener del estudio}

\subsection{Formación de recursos humanos}

Se graduará durante el segundo año a un alumno de licenciatura.

\subsection{Comunicación de resultados}

\subsubsection{Comunicación en congresos}

Se buscará participar a lo largo del segundo año en un congreso nacional con una
contribución de tipo oral más una de tipo poster además de en un congreso internacional con
una contribución de tipo oral más una de tipo poster.

\subsubsection{Comunicación en artículos}

Se publicarán a lo largo de los dos años, y como está programado en el
cronograma, 4 artículos en revistas internacionales indexadas de alto impacto: 1
artículo de revisión y 3 artículos de investigación original.

\subsection{Utilidad del trabajo de investigación}

El proyecto está dirigido a obtener la información relevante que nos permita
\red{determinar las regiones más apropiadas de las proteínas constitutivas de
  EBV susceptibles de ser ``drogadas'' con un posible efecto final inhibitorio.
  Estos resultados son esenciales para sentar las bases} de un proyecto de
diseño y optimización de moléculas con potencial farmacológico como antivirales
de EBV.

\section{Desglose de presupuesto}
\emph{Ver Formatos Adjuntos.}


\newpage

\addcontentsline{toc}{section}{Referencias}
\bibliographystyle{myunsrt_wtitle}
\bibliography{/home/diego/MendeleyBibtex/Mis_Escritos-Mis_Propuestas-Fondos_Federales_2017}

\end{document}

%  LocalWords:  Barr
